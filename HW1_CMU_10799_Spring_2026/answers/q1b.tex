Based on visual exploration of the dataset grid and analysis of the attribute distributions, this subset was clearly filtered from full CelebA using the following criteria:

\begin{itemize}
    \item \textbf{No Beards (100\%)}: Every single image in the subset has the ``No\_Beard'' attribute set to 1. This was likely the primary filtering criterion to ensure clean facial features without facial hair.
    
    \item \textbf{Young Faces (73.9\%)}: The vast majority of faces are young (approximately 74\% marked as young). This suggests the dataset was filtered to focus on younger individuals, likely to avoid facial wrinkles and age-related features.
    
    \item \textbf{Balanced Gender Distribution (55.5\% Male)}: While not extremely filtered, there's a roughly balanced gender representation with a slight male bias, unlike raw CelebA which can be heavily skewed.
    
    \item \textbf{Clean, High-Quality Faces}: The subset avoids extremely blurry images and appears to be curated for image quality.
    
    \item \textbf{Diverse Expressions and Features}: Images include a mix of smiling (44\%) and non-smiling faces, with diverse hair colors and facial features, allowing the model to learn varied facial characteristics.
\end{itemize}

\textbf{Why these criteria?} These filtering decisions create a more challenging and interesting learning task for a diffusion model. By removing beards entirely and focusing on young faces, the dataset creates a cohesive domain that is neither too narrow (which would make the task trivial) nor too diverse (which would require enormous data). The subset represents a natural distribution that's suitable for benchmarking modern generative models while avoiding the extreme heterogeneity that makes full CelebA difficult to model well.
