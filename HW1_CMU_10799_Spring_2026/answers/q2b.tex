\textbf{Data Augmentation Strategy:} Given the filtered nature of our dataset (young, beardless faces), I propose a \textbf{conservative set of augmentations}:

\begin{enumerate}
    \item \textbf{Random Horizontal Flips (50\% probability)}: This is safe because facial symmetry is important for face generation, but horizontal flips preserve all the key attributes (young, no beard, etc.).
    
    \item \textbf{Small Random Rotations (-10° to +10°)}: Very slight rotations can help the model learn slight head poses while preserving the fundamental attributes. Too large rotations would be unrealistic.
    
    \item \textbf{Mild Color/Brightness Jittering}: Small adjustments to brightness and contrast simulate different lighting conditions, which is common in real photography and helps with generalization.
    
    \item \textbf{Avoid Aggressive Transformations}: I would \textbf{avoid}:
    \begin{itemize}
        \item Random crops (loses important facial information)
        \item Large rotations or perspective transforms (creates unrealistic face angles)
        \item Aggressive color shifts (could alter skin tone representation)
    \end{itemize}
\end{enumerate}

\textbf{Why?} Face generation requires preserving fine spatial relationships (eyes, nose, mouth alignment). Our filtered dataset is already somewhat limited (~64K images), so overly aggressive augmentations could introduce artifacts or make the training task harder. Conservative augmentations help with generalization and lighting invariance without distorting facial geometry.
