\vspace{1em}
\noindent{\Large\bfseries\textcolor{cmured}{Introduction}}
\vspace{0.5em}
\hrule
\vspace{0.8em}

Welcome to your first homework in CMU 10799 Diffusion \& Flow Matching!

In this homework, you'll implement a Denoising Diffusion Probabilistic Model (DDPM)~\citep{ho2020ddpm} from scratch and train it to generate realistic face images. But more than just "getting it to work," we want you to develop the intuition and debugging skills that researchers use every day.
The structure of this homework mirrors a pseudo research workflow:

\begin{enumerate}[
    leftmargin=*,
    label=\textbf{\textcolor{cmured}{\arabic*.}},
    itemsep=0.2em
]
    \item Understand your data: Before writing any model code, explore what you're trying to generate
    \item Build intuition: Use the 1D playground to see diffusion in action (optional but recommended)
    \item Implement: Write the core DDPM algorithm and U-Net architecture
    \item Debug: Learn to diagnose common failure modes
    \item Experiment: Run ablations to understand design choices
    \item Reflect: Document what you learned and what you'd try next
\end{enumerate}

This homework is AI-friendly. You may use any AI coding assistants, chatbots, or reference implementations. You may also use any other resources that you can find on the Internet. At the end of the homework, you'll document what resources you used.