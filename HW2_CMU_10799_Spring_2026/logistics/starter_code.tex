
\hwpart{0}{Setup your codebase (0 point)}
You'll continue using the same codebase from HW1. Your DDPM implementation should still work and you'll need it for this homework.
\noindent\textbf{\textcolor{cmured}{What You'll Implement:}}
\vspace{0.6em}

\begin{tcolorbox}[
    colback=cmulight,
    colframe=cmugray,
    boxrule=0.5pt,
    arc=2pt,
    left=8pt,
    right=8pt,
    top=6pt,
    bottom=6pt
]
    \centering
    \renewcommand{\arraystretch}{1.2}
    \begin{tabular}{ll}
    \toprule
    \textbf{File} & \textbf{What to implement} \\
    \midrule
    \texttt{src/methods/flow\_matching.py} & Create your own flow matching implementation \\
    \texttt{src/methods/ddpm.py} & Add DDIM sampling \\
    \texttt{configs/} & Add configs for flow matching \& DDIM \\
    \texttt{scripts/} & Add scripts for flow matching \& DDIM \\
    \texttt{train.py} & Add flow matching training options \\
    \texttt{sample.py} & Add flow matching \& DDIM sampling options \\
    \bottomrule
    \end{tabular}
\end{tcolorbox}

\vspace{2em}
\noindent\textbf{\textcolor{cmured}{Hint for Flow Matching:}}
\vspace{0.6em}

\noindent Your \texttt{\textcolor{cmured}{flow\_matching.py}} should follow the same structure as your \texttt{\textcolor{cmured}{ddpm.py}}:
\begin{itemize}[leftmargin=*]
    \item A \texttt{\textcolor{cmured}{FlowMatching}} class that inherits from \texttt{\textcolor{cmured}{BaseMethod}}
    \item \texttt{\textcolor{cmured}{compute\_loss(x\_0, ...)}} — returns training loss given a batch of real images
    \item \texttt{\textcolor{cmured}{sample(batch\_size, image\_shape, ...)}} — generates new images starting from noise
\end{itemize}

\noindent The core algorithm is different (and simpler!) but the interface should feel familiar.

\vspace{1em}
\noindent\textbf{\textcolor{cmured}{Compute Budget:}}
\vspace{0.6em}

You have \textbf{\$500 in \href{https://modal.com/}{Modal} credits} for the entire course (all 4 homework). Budget wisely!
