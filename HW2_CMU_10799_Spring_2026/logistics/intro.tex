\vspace{1em}
\noindent{\Large\bfseries\textcolor{cmured}{Introduction}}
\vspace{0.5em}
\hrule
\vspace{0.8em}

Welcome to Homework 2! In HW1, you built a DDPM~\citep{ho2020ddpm} model that learns to denoise images step by step. Now you'll \textbf{implement Flow Matching}~\citep{lipman2023flow}, a simpler and often more efficient framework that learns a velocity field to transport noise to data along straight paths.

In addition to flow matching, you will also \textbf{implement DDIM}~\citep{song2020denoising}, a deterministic sampler that lets your existing DDPM model generate images in far fewer steps. Explore and compare flow matching and DDIM with DDPM sampling from HW1!

By the end of this homework, you'll have two working generative models (DDPM and Flow Matching) and will \textbf{choose your specialization track} for the rest of the course.

This homework is AI-friendly. Same rules as HW1. You may use any AI coding assistants, chatbots, or reference implementations. You may also use any other resources that you can find on the Internet. At the end of the homework, you'll document what resources you used.