
\hwpart{I}{Implement Flow Matching (25 points)}
Now it's time to build your flow matching model! Take a good look at what you have built so far and start building again! Remember, you are free to add, delete and modify any and all parts of the starter code to fit your preference.

\question{Building Intuition}{optional, 0}

You can still leverage the same strategies from HW1 to develop some intuitions of the algorithm with some toy experiments before diving into the full scale training.

\subq{a}{0} You can still use the Jupyter notebook \texttt{\textcolor{cmured}{01\_1d\_playground.ipynb}} to experiment with Flow Matching on 1D toy data. This notebook contains some options of mixture of Gaussians and their visualizations, you can try out your algorithm first in this playground.

\subq{b}{0} As with HW1, you can use the \texttt{\textcolor{cmured}{-\--overfit-single-batch}} flag to sanity check your implementation before full-scale training.

\question{Implementation}{25}

Now let's implement flow matching! Use the same codebase you have from HW1 as the skeleton, and your job is to fill in the core algorithm.
Remember: this is your codebase. Feel free to modify any part of the starter code to fit your preferences. Add helper functions, reorganize files, change the config structure... whatever helps. The only requirement is that your final code runs and produces good results.

\subq{a}{5} Train your flow matching model to convergence. You may use either the provided configs or your own. \textbf{You should use roughly the same model architecture, training iterations and batch size as your DDPM model.} Report:
\begin{enumerate}
    \item Model size
    \item Batch size and total training iterations
    \item Training loss curve
    \item Compute cost (GPU hours)
    \item Number of sampling steps
\end{enumerate}
\answerbox{}{
% Your answer here
\vspace{5.5cm} % comment this line out after you put in your answers
}
\includeanswer{q4a}

\subq{b}{10} Generate a grid of 16 samples from your trained model. Include this grid below.
\answerbox{}{
% Your answer here
\vspace{10cm} % comment this line out after you put in your answers
}
\includeanswer{q4b}

\subq{c}{10} Evaluate your model KID with 1k samples and report your KID score (both mean and std). You should be able to obtain KID < 0.005 with single L40S GPU training for a few hours.
\answerbox{}{
% Your answer here
}
\includeanswer{q4c}

\subq{d}{0} Provide a zip file of your code through Gradescope ``Homework 2 Code'' assignment. Make sure your code is runnable because we may run it to verify your results, and do not include any large files such as model checkpoints or dataset files. If you do not provide a zip file for your code, you will receive 0 point on Part I, Part III and Part IV of this homework.
\includeanswer{q4d}