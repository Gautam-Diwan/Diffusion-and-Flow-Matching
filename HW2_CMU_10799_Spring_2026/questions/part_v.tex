\hwpart{V}{Track Selection (10 points)}

For the rest of the course, you'll specialize in one of three tracks. Your choice determines what you'll implement in HW3 and HW4.

\vspace{1em}
\noindent\textbf{\textcolor{cmured}{The Three Tracks:}}
\vspace{0.5em}

\begin{itemize}[leftmargin=*]
    \item \textbf{Fidelity}: Best image quality
    
    Your goal is to generate the highest quality images possible. This track is all about pushing the limits of sample quality through architecture improvements, better training recipes, advanced samplers, or any other techniques you can find. You'll measure success primarily through KID/FID scores, but visual quality matters too. Example directions include: trying different model architectures (e.g., DiT, U-ViT), experimenting with training techniques (e.g., noise schedule tuning), or implementing advanced samplers.
    
    \textit{Evaluation Metric:} KID/FID $\downarrow$
    
    \vspace{0.5em}
    
    \item \textbf{Controllability}: User control over generation
    
    Your goal is to control what the model generates. This is the most open-ended track: you have creative freedom to define what ``control'' means to you. Some ideas: conditioning on attributes (e.g., generate smiling faces, faces with glasses), text-to-image generation, image editing, interpolation in latent space, style transfer, or anything else you can imagine. Be creative! But remember: with great power comes great responsibility. Since this track is open-ended, you'll also need to define your own evaluation metrics that make sense for your chosen form of control.
    
    \textit{Evaluation Metric:} You define it! (e.g., CLIP score, attribute classifier accuracy, user study, or custom metrics that fit your approach)
    
    \vspace{0.5em}
    
    \item \textbf{Speed}: Fast inference
    
    Your goal is to generate good images in as few sampling steps as possible. This track explores the tradeoff between speed and quality: can you match your baseline quality with 10x fewer steps? 100x fewer? Or even with single step sampling? You'll explore techniques like distillation, consistency models, better ODE solvers, or learned step-size schedules. Success is measured by how few steps you need to reach a target KID threshold.
    
    \textit{Evaluation Metric:} Number of steps to reach KID $<$ threshold (e.g., KID $<$ 0.005)
    
\end{itemize}

\question{Track Selection}{10}

\subq{a}{2} Which track do you choose?

\answerbox{}{
% Your answer here
\vspace{1cm} % comment this line out after you put in your answers
}
\includeanswer{q8a}

\subq{b}{3} Why does this track interest you? What draws you to this particular challenge?

\answerbox{}{
% Your answer here
\vspace{3.5cm} % comment this line out after you put in your answers
}
\includeanswer{q8b}

\subq{c}{3} What's one thing you're uncertain about or curious to explore in your chosen track?

\answerbox{}{
% Your answer here
\vspace{3.5cm} % comment this line out after you put in your answers
}
\includeanswer{q8c}

\subq{d}{2} What resources (papers, repos, tutorials) have you found that might help with your chosen track? List at least 2.

\answerbox{}{
% Your answer here
\vspace{3cm} % comment this line out after you put in your answers
}
\includeanswer{q9b}
