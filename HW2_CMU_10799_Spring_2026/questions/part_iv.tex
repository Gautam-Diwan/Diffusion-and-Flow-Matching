\hwpart{IV}{Ablation Study (25 points)}

In this part of the homework, we will explore how the number of sampling steps affects sample quality for both Flow Matching and DDIM.

\question{Sampling Steps Ablation}{25}

You now have three ways to generate images: DDPM (1000 steps from HW1), DDIM (fewer steps, same trained DDPM model), and Flow Matching (Euler integration). Let's see how they compare.

\subq{a}{15} Report KID scores (mean and std) for Flow Matching and DDIM at different step counts. Fill in the table below:

\begin{center}
\begin{tabular}{|c|c|c|}
\hline
\textbf{Steps} & \textbf{Flow Matching KID} & \textbf{DDIM KID} \\
\hline
1 & & \\
\hline
5 & & \\
\hline
10 & & \\
\hline
50 & & \\
\hline
100 & & \\
\hline
200 & & \\
\hline
1000 &  & \\
\hline
\end{tabular}
\end{center}

\subq{b}{10} Based on your results, compare flow matching, DDIM, and your original DDPM from HW1:
\begin{enumerate}
    \item At what step count does each method start producing reasonable samples?
    \item For each method, what is the minimum number of steps needed to achieve similar quality to your DDPM with 1000 steps from HW1?
    \item How do you compare flow matching and DDIM with DDPM at 100 steps? What about at 1000 steps?
    \item If you wanted the best sample quality regardless of speed, which method would you choose? If you wanted fast generation with acceptable quality, which would you choose?
\end{enumerate}
 

\answerbox{}{
% Your answer here
\vspace{6cm} % comment this line out after you put in your answers
}
\includeanswer{q7b}